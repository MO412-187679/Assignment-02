\section{Image (a)}

\begin{wrapfigure}{l}{0.5\textwidth}
    \centering
    \begin{tikzpicture}[draw=darkgray,text=darkgray, align=center, node distance=4cm]
        \tikzstyle{every node}=[inner sep=0pt];

        \node (v1) [label=above:{\Large $v_1$}] {};
        \node (v2) [label=left:{\Large $v_2$}, below left of = v1] {};
        \node (v3) [label=right:{\Large $v_3$}, below right of = v1] {};
        \node (v4) [label=below:{\Large $v_4$}, below right of = v2] {};

        \path (v1.center) edge (v2.center) edge (v3.center);
        \path (v4.center) edge (v2.center) edge (v3.center);
        \path (v2.center) edge (v3.center);
    \end{tikzpicture}

    \caption{Image \texttt{a.} from question with labelled nodes.}
    \label{fig:graph-a}
\end{wrapfigure}

A \textbf{trail} is a sequence of \textit{distinct} edges such that any consecutive edges are incident to a common vertex. When the sequence visits every edge of a graph, it is called an \textbf{Euler trail}. Following such trail with a pencil would lead us to draw the graph with the specified requirements.

To find if a graph has an Euler trail, however, we need \nameref{thm:euler-circuit}, which is stated in terms of a \textbf{circuit}, that is a trail that ends at the same vertex it started. From there we have \cref{thm:euler-trail}, which won't be proved here, but the main argument behind it is:

If we connect the two vertices with odd degree by an edge $e := x y$, we get an even graph. Since this new graph has an Euler circuit $W := u W_1 x e y W_2 u$ and the only edge missing is $e$, the trail $T := x W_1 u W_2 y$ still is eulerian for the original graph.

~

\begin{theorem}[Euler's Theorem] \label{thm:euler-circuit}
    A connected graph has an Euler circuit if and only if every vertex has even degree.
\end{theorem}

\begin{proposition} \label{thm:euler-trail}
    A connected graph has an Euler trail if and only if exactly zero or two vertices have odd degree.
\end{proposition}

\subsection{Answer}

    Finally, by \cref{thm:euler-trail}, we can see that the graph in \cref{fig:graph-a} can be drawn without raising the pencil and without repeating any lines. This is because only $v_2$ and $v_3$ have odd degree ($\deg(v_2) = 3 = \deg(v_3)$).

    Furthermore, considering the argument for \cref{thm:euler-trail}, we must start the drawing at one of the two odd vertices, $v_2$ or $v_3$.
