\def\homeworkname{Koenigsberg}
\documentclass[assignment = 2]{homework}

\usepackage{caption, subcaption, pdfpages, float}
\usepackage{graphics, wrapfig, pgf, graphicx}
\usepackage{enumitem}


% pacotes para importar código
\usepackage{caption, booktabs}
\usepackage[section, newfloat]{minted}
\definecolor{sepia}{RGB}{252,246,226}
\setminted{
    bgcolor = sepia,
    style   = pastie,
    frame   = leftline,
    autogobble,
    samepage,
    python3,
    breaklines
}
\setmintedinline{
    bgcolor={}
}

% ambientes de códigos de Python
\newmintedfile[pyinclude]{python3}{}
\newmintinline[pyline]{python3}{}
\newcommand{\pyref}[2]{\href{#1}{\texttt{#2}}}

% \SetupFloatingEnvironment{listing}{name=Código}
% \captionsetup[listing]{position=below,skip=-1pt}

\usepackage{csquotes}
\usepackage[
    style    = verbose-ibid,
    autocite = footnote,
    notetype = foot+end,
    backend  = biber
]{biblatex}
\addbibresource{references.bib}
\usepackage[section]{placeins}

\usepackage[hidelinks]{hyperref}
\usepackage[noabbrev, nameinlink]{cleveref}
\hypersetup{
    pdftitle  = {MO412/MC908 - Assignment 1},
    pdfauthor = {Tiago de Paula - 187679}
}

\newcommand{\textref}[2]{
    \hyperref[#2]{#1 \ref*{#2}}
}

\renewcommand{\vec}[1]{\mathbf{#1}}

\DeclareMathOperator{\round}{round}

\usepackage{import, multirow}
\usepackage{pgf, tikz}
\usetikzlibrary{matrix}
\usetikzlibrary{positioning}
\usetikzlibrary{automata}
\usetikzlibrary{shapes}

\usepackage{wrapfig}
\usepackage{booktabs}

\newenvironment{kmatrix}[1][1.3cm]{
    \begin{tikzpicture}[node distance=0cm]
        \tikzset{square matrix/.style={
                matrix of nodes,
                column sep=-\pgflinewidth, row sep=-\pgflinewidth,
                nodes={draw,
                    minimum height=#1,
                    anchor=center,
                    text width=#1,
                    align=center,
                    inner sep=0pt
                },
            },
            square matrix/.default=#1
        }
}{
    \end{tikzpicture}%
}

\newcommand*{\Scale}[2][4]{\scalebox{#1}{\ensuremath{#2}}}%

\newcommand{\red}[1]{\textcolor{red}{\textbf{#1}}}
\def\qm{?}

\begin{document}
    \pagestyle{main}

    Which of the graphs in the attached figure can be drawn without raising your pencil from the paper, and without drawing any line more than once? Why? For the graphs that can be drawn as described, state how many starting points are possible.

    \section{Image (a)}

\begin{wrapfigure}{l}{0.5\textwidth}
    \centering
    \begin{tikzpicture}[draw=darkgray, text=darkgray, align=center, node distance=4cm]
    \tikzstyle{every node}=[inner sep=0pt];

    \node (v1) [label=above:{\Large $v_1$}] {};
    \node (v2) [label=left:{\Large $v_2$}, below left of = v1] {};
    \node (v3) [label=right:{\Large $v_3$}, below right of = v1] {};
    \node (v4) [label=below:{\Large $v_4$}, below right of = v2] {};

    \path (v1.center)
        edge (v2.center)
        edge (v3.center);
    \path (v4.center)
        edge (v2.center)
        edge (v3.center);
    \path (v2.center)
        edge (v3.center);
\end{tikzpicture}


    \caption{Image \texttt{a.} from question with labelled nodes.}
    \label{fig:graph-a}
\end{wrapfigure}

A \textbf{trail} is a sequence of \textit{distinct} edges such that any consecutive edges are incident to a common vertex. When the sequence visits every edge of a graph, it is called an \textbf{Euler trail}. Following such trail with a pencil would lead us to draw the graph with the specified requirements.

To find if a graph has an Euler trail, however, we need \nameref{thm:euler-circuit}, which is stated in terms of a \textbf{circuit}, that is a trail that ends at the same vertex it started. From there we have \cref{thm:euler-trail}, which won't be proved here, but the main argument behind it is:

If we connect the two vertices with odd degree by an edge $e := x y$, we get an even graph. Since this new graph has an Euler circuit $W := u W_1 x e y W_2 u$ and the only edge missing is $e$, the trail $T := x W_1 u W_2 y$ still is eulerian for the original graph.

~

\begin{theorem}[Euler's Theorem] \label{thm:euler-circuit}
    A connected graph has an Euler circuit if and only if every vertex has even degree.
\end{theorem}

\begin{proposition} \label{thm:euler-trail}
    A connected graph has an Euler trail if and only if exactly zero or two vertices have odd degree.
\end{proposition}

\subsection{Answer} \label{sec:graph-a}

    Finally, by \cref{thm:euler-trail}, we can see that the graph in \cref{fig:graph-a} can be drawn without raising the pencil and without repeating any lines. This is because only $v_2$ and $v_3$ have odd degree ($\deg(v_2) = 3 = \deg(v_3)$).

    Furthermore, considering the argument for \cref{thm:euler-trail}, we must start the drawing at one of the two odd vertices, $v_2$ or $v_3$.

~

    \section{Image (b)}

\noindent
\begin{minipage}[t]{0.49\textwidth}

    For \cref{fig:graph-b}, we have nodes $v_1$, $v_2$, $v_3$ and $v_4$ with degree odd ($\deg(v_i) = 3$ for $i = 1$, $2$, $3$ and $4$), therefore the graph does \textbf{not} have an Euler trail and cannot be drawn without raising the pencil or repeating a line.

\end{minipage}%
\begin{minipage}[t]{0.5\textwidth}

    \begin{figure}[H]
        \centering
        \begin{tikzpicture}[draw=darkgray,text=darkgray, align=center, node distance=3cm]
    \tikzstyle{every node}=[inner sep=0pt];

    \node (v5) [label=above right:{\Large $v_5$}] {};
    \node (v1) [label=above:{\Large $v_1$}, above of = v5] {};
    \node (v2) [label=left:{\Large $v_2$}, left of = v5] {};
    \node (v3) [label=right:{\Large $v_3$}, right of = v5] {};
    \node (v4) [label=below:{\Large $v_4$}, below of = v5] {};

    \path (v1.center)
        edge (v2.center)
        edge (v3.center);
    \path (v4.center)
        edge (v2.center)
        edge (v3.center);
    \path (v5.center)
        edge (v1.center)
        edge (v2.center)
        edge (v3.center)
        edge (v4.center);
\end{tikzpicture}


        \caption{Image \texttt{b.} with labelled nodes.}
        \label{fig:graph-b}
    \end{figure}

\end{minipage}

    \section{Image (c)}

\noindent
\begin{minipage}[t]{0.6\textwidth}

    \begin{figure}[H]
        \centering
        \begin{tikzpicture}[draw=darkgray, text=darkgray, align=center, xscale=1, yscale=sqrt(3)/2, scale=2]
    \tikzstyle{every node}=[inner sep=0pt];

    \node at ( 0.0,  2)  (a) [label=above:{\Large\hypertarget{fig:graph-c:v1}{$v_{1}$}}] {};
    \node at (-1.5,  1)  (b) [label=above left:{\Large\hypertarget{fig:graph-c:v6}{$v_{6}$}}] {};
    \node at (-0.5,  1)  (c) [label=below right:{\Large\hypertarget{fig:graph-c:v7}{$v_{7}$}}] {};
    \node at ( 0.5,  1)  (d) [label=below left:{\Large\hypertarget{fig:graph-c:v8}{$v_{8}$}}] {};
    \node at ( 1.5,  1)  (e) [label=above right:{\Large\hypertarget{fig:graph-c:v2}{$v_{2}$}}] {};
    \node at (-1.0,  0)  (f) [label=right:{\Large\hypertarget{fig:graph-c:v12}{$v_{12}$}}] {};
    \node at ( 1.0,  0)  (g) [label=left:{\Large\hypertarget{fig:graph-c:v9}{$v_{9}$}}] {};
    \node at (-1.5, -1)  (h) [label=below left:{\Large\hypertarget{fig:graph-c:v5}{$v_{5}$}}] {};
    \node at (-0.5, -1)  (i) [label=above right:{\Large\hypertarget{fig:graph-c:v11}{$v_{11}$}}] {};
    \node at ( 0.5, -1)  (j) [label=above left:{\Large\hypertarget{fig:graph-c:v10}{$v_{10}$}}] {};
    \node at ( 1.5, -1)  (k) [label=below right:{\Large\hypertarget{fig:graph-c:v3}{$v_{3}$}}] {};
    \node at ( 0.0, -2)  (l) [label=below:{\Large\hypertarget{fig:graph-c:v4}{$v_{4}$}}] {};

    \path (c.center)
        edge (a.center)
        edge (b.center)
        edge (d.center);
    \path (d.center)
        edge (a.center)
        edge (e.center)
        edge (g.center);
    \path (g.center)
        edge (e.center)
        edge (k.center)
        edge (j.center);
    \path (j.center)
        edge (k.center)
        edge (l.center)
        edge (i.center);
    \path (i.center)
        edge (l.center)
        edge (h.center)
        edge (f.center);
    \path (f.center)
        edge (h.center)
        edge (b.center)
        edge (c.center);
\end{tikzpicture}


        \caption{Image \texttt{c.} with labelled nodes.}
        \label{fig:graph-c}
    \end{figure}

\end{minipage}%
\begin{minipage}[t]{0.39\textwidth}

    Note that every outer vertex here ($\hyperlink{fig:graph-c:v1}{v_1}, \ldots, \hyperlink{fig:graph-c:v6}{v_6}$) has degree 2 and every inner node ($\hyperlink{fig:graph-c:v7}{v_7}, \ldots, \hyperlink{fig:graph-c:v12}{v_{12}}$) has degree 4 therefore, by \nameref{thm:euler-circuit}, there is an Euler circuit and the graph can be drawn as described.

    ~

    \indent Since a circuit is just a collection of cycles, we can start the eulerian trail from any vertex and we will stop at the same vertex. In fact, we can start at any point in the drawing, not only the labelled vertices, by just creating a vertex there with degree two.

\end{minipage}

    \section{Image (d)}

\begin{table}[H]
    \caption{Degrees of each vertex of \cref{fig:graph-d}.}
    \label{tab:graph-d}

    \centering
    \begin{tabular}{ccc}
        \toprule
        \toprule
            Vertex & Degree & Is Odd? \\
        \midrule
            $\hyperlink{fig:graph-d:v1}{v_1}$ & 2 & --- \\
            $\hyperlink{fig:graph-d:v2}{v_2}$ & 2 & --- \\
            $\hyperlink{fig:graph-d:v3}{v_3}$ & 6 & --- \\
            $\hyperlink{fig:graph-d:v4}{v_4}$ & 6 & --- \\
            $\hyperlink{fig:graph-d:v5}{v_5}$ & 3 & Yes \\
            $\hyperlink{fig:graph-d:v6}{v_6}$ & 1 & Yes \\
        \bottomrule
        \bottomrule
    \end{tabular}
\end{table}

\begin{wrapfigure}{r}{0.5\textwidth}
    \centering
    \begin{tikzpicture}[draw=darkgray, text=darkgray, align=center, xscale=1, yscale=2, scale=2]
    \tikzstyle{every node}=[inner sep=0pt];

    \node at ( 0.0, 1) (v1) [label=above:{\Large $v_1$}] {};
    \node at (-1.5, 0) (v2) [label=below left:{\Large $v_2$}] {};
    \node at (-0.5, 0) (v3) [label=below left:{\Large $v_3$}] {};
    \node at ( 0.0, 0) (vc) {};
    \node at ( 0.5, 0) (v4) [label=below right:{\Large $v_4$}] {};
    \node at ( 1.5, 0) (v5) [label=right:{\Large $v_5$}] {};
    \node at ( 1.5, -0.7) (v6) [label=below:{\Large $v_6$}] {};

    \draw (v1.center)
        edge (v2.center);
    \draw (v3.center)
        edge (v2.center)
        edge (v4.center);
    \draw (v5.center)
        edge (v1.center)
        edge (v4.center)
        edge (v6.center);
    \draw (vc.center) ellipse (0.5 and 0.8);
    \draw (vc.center) ellipse (0.5 and 0.5);
\end{tikzpicture}


    \caption{Image \texttt{d.} with labelled nodes.}
    \label{fig:graph-d}
\end{wrapfigure}

Just like with \hyperref[sec:graph-a]{image (a)}, we have exactly two odd vertices, $\hyperlink{fig:graph-d:v5}{v_5}$ and $\hyperlink{fig:graph-d:v6}{v_6}$, in this case. By \cref{thm:euler-trail}, this proves the graph can be drawn as required. As before, both vertices can be used as the starting point for the drawing.

\end{document}
